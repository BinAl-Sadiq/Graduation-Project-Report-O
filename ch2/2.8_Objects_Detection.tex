
\section{Objects Detection}
A goal of computer vision that tasks a device to identify an object via unique attributes such it's silhouette and perceived shape while accounting illumination and the viewpoint of it and classify it under known predefined categories (such as a car or a trash can) then put a rectangle called a bounding box indicating the name of an object and it's size relative to the  perspective of the pupil of the device. Object Detection often involves training a model on a large data-set to help it recognize a pattern based on the logical facts it has and match objects with that set.

\subsection{General Structure of a Model}
Each of the models we're about to show follow a general structure in processing, first, the model asserts a location of the region/boundary of an object by taking all the points detected with a data-set that has been collected through rigorously taking pictures, classifying them under a category which forms a data-set to adjust values of the model (a process that's called training), then the models go inside the boundary box in search of an object; this can be done in a variety of ways, one of them is with using a filter (linked to a Convoluted Neural Network) that contains a part or a whole that refers to a unique feature of the object (it's face, body, etc.) and let it output a score of how well it matched with the target chosen. There can be more than filter also as depicted in some of these models, if that's the case, then all of their results are combined in a single score 2D matrix, then processed by a Support Vector Machine (SVM) which as the name implies, acts as an additional categorical filter based on the values found in the vector matrix, then it gets slotted into one of many categories the model has which classified with a unique name/id. There are various models to accomplish such a task but we narrowed them down to a few key ones whom show the most potential in our use-case.
These are: Deformable Parts Model (DPM), and YOLO (You Only Look Once).
We will proceed to look at each and highlight their key features from a non-mathematical functional view below.

\subsection{Deformable Parts Model}
The main idea of this method is objects are the sum of their parts, with separate dedicated filters for each of the machines employed to help the device capture accurate guesses, 
it uses an enriched Dalal-Triggs model (the root filter) which then incorporated in a dot product of weights from the other filters, these filters are:
\begin{itemize}
	\item \textbf{coarse root filter:} tasked to detect the silhouette of the object without taking
	into account the parts.
	\item \textbf{parts filter:} each distinguishable feature of the silhouette.
	\item \textbf{Spatial filter model:} Indicates the location of each part in relation to the root.
\end{itemize}
The coarse root filter is the main bone of the other filters, and it what mostly determines a guess, if that wasn't enough however, the two next filters come into play to further detert.
 Each of these filters run in tandem to produce a cumulative guess matrix which then get fed to several SVMs for categorization. The model is complex because it has 3 different filters which need to be trained and refined, hampering it's real-time score.
See reference [num] for more details.

\subsection{You Only Look Once (YOLO)}
Opposite to the above method, YOLO only views the image once by basing all of the calculations on a single image, eliminating the need of multiple, parallel processes running together which ultimately reduces the complexity of a single calculation. The algorithm takes a two-dimensional input image from a given source, overlay an evenly-spaced square with dimensions n x n on each of the two axis (not a large number), then take the  bounding boxes present in the grid and the confidence map, where it determines which of the objects in the cells are and categorize them to finally be combined, resulting in a list of thing it detected in a heatmap, a score which it then combines with the rest to get a final guess. It's capable of being put in real-time applications thanks to it only necessitating a single CNN to work efficiently and effectively. 
However, it's not without limitations, each cell of the grid can only have one class, so if two object of different classes emerge in a cell, the algorithm will have difficulties detecting them. It also doesn't work in environments dense with objects of a certain class due to 
See reference [num] for more details.
YOLO has a number of different versions since the paper had been written, each of them improve the performance significantly, the version present as of now is YOLOv10, a package readily available for developers.
\section{Application Programming Interface}

An Application Programming Interface (API) is a set of rules that allows different software applications to communicate with each other. It defines the methods and data formats that applications can use to request and exchange information, simplifying the process for developers to build compatible software. For instance, when using a social media login to access a new app, an API handles the authentication process.

APIs are essential in various technology domains, including web and mobile applications, where they enable diverse services to work together seamlessly. For example, a travel booking website may utilize APIs to aggregate data from various airline and hotel services, allowing users to compare prices and availability in one place. Similarly, a weather app may leverage APIs to retrieve real-time weather data from different sources, providing accurate forecasts based on user location \cite{ibm2024}.


\textbf{Base URL:} The base URL is the starting point for accessing an API. It usually includes the protocol (like HTTP or HTTPS) and the domain name where the API is hosted. For example, if the base URL is \texttt{https://api.example.com}, all API requests will start from this address.

\textbf{Endpoints:} Endpoints are specific paths added to the base URL that direct requests to particular resources or functions within the API. For example, if you want to access user data, you might use an endpoint like \texttt{/users}. The full URL for this request would be \texttt{https://api.example.com/users}.

\textbf{HTTP Methods:} APIs commonly use HTTP methods to specify the action to be performed. The most common methods are:
\begin{itemize}
	\item \texttt{GET}: Retrieve data from the server.
	\item \texttt{POST}: Send new data to the server.
	\item \texttt{PUT}: Update existing data on the server.
	\item \texttt{DELETE}: Remove data from the server.
\end{itemize}
 
 \textcolor{blue}{
 	\begin{itemize}
 		\item ADD frameworks example, including what will be used in the next semester.
 \end{itemize}}


\section{Landmarks (Natural \& Artificial)}

Landmarks can be categorized as either natural or artificial. Natural landmarks
are formed by nature, such as mountains, rocks, trees, or any other natural
formations. While on the other hand, artificial landmarks are human-made
structures such as traffic signs, statues, and QR codes. both categorizes can be used in various important fields in computer science. For example, computer vision algorithms leverage landmarks by detecting, identifying, and tracking them to determine the precise location of an object, create a 3D map of the surrounding environment, or to monitor the state of an object.

While both categorizes can be very useful, natural landmarks are much harder to recognize, very diversified and do not have uniform shapes, difficult to customize, and they do not encode data. All of these characteristics can be opposite in the artificial landmarks, which make them better in terms of control, precision, reliability, and adaptability. Also there are a lot of libraries that support encoding, detecting, decoding, and tracking artificial landmarks. There are also a wide variety of artificial markers such as QR, ArUco, Topotag. More information on the other kinds of markers are present on reference \cite{Fiducial2021}.






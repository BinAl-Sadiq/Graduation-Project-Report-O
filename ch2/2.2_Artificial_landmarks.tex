\section{Landmarks}

Landmarks are essential in navigation, tracking, and mapping, helping to determine positions within an environment. They can be categorized into two types: natural and artificial.

\subsection{Natural Landmarks}

Natural landmarks are features formed by nature or naturally present in an environment, such as mountains, rocks, trees, stairs, or doors. In computer science, they can be used in computer vision algorithms to help determine object locations, create 3D maps, or monitor object states. However, natural landmarks are more difficult to recognize due to their diversity, lack of uniform shapes, and inability to encode data.

\subsection{Artificial Landmarks}

Artificial landmarks, on the other hand, are human-made structures like traffic signs, statues, and QR codes. These landmarks offer several advantages over natural ones in computer vision applications due to their uniformity, ease of recognition, ability to encode data, and customization options. Artificial landmarks are more reliable, precise, and adaptable, making them better suited for controlled environments. Furthermore, many libraries support the encoding, detection, decoding, and tracking of artificial landmarks. Examples include various markers such as QR codes, Aruco markers, and Topotags, which are widely used in different computer science fields. More details on different types of markers can be found in reference \cite{Fiducial2021}.



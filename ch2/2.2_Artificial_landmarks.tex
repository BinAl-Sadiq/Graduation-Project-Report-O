\section{Artificial Landmarks}

Landmarks are essential in various fields of computer science, such as robotics, localization, and computer vision. They can be classified as either natural or artificial. Natural landmarks are features formed by nature—like mountains, rocks, and trees. While these are useful in applications like self-driving cars and geographic information systems (GIS), they present challenges for localization tasks due to their diversity, lack of uniform shapes, difficulty in recognition, and inability to encode data.

Artificial landmarks, on the other hand, are human-made structures such as buildings, traffic signs, statues, and QR codes. These landmarks excel in localization tasks because they offer features that are easy to detect, recognize, and customize. They often have uniform shapes and can encode data, making them highly effective for precise localization and object tracking. There are wide variety of artificial markers such as QR, Arcuo, and Topotag. For more information, see reference \cite{kalaitzakis2021}.
	

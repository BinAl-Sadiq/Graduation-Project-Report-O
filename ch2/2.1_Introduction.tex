\section{Introduction}
The world isn't fair for the visually impaired, plenty of normal, everyday routes become hazardous without any way to see them, stairs, crowds, elevators, stores and plenty of stuff they're oblivious about existing unless someone guides one to them; even regular protrusions are tripping hazards and this is prevalent in  most buildings where safety standards aren't enforced which accounts for most residential blocks in saudi arabia.
There's also a rising concern of people with no family acquaintances or vacant assistants which gets more prevalent year by year thanks to the rise of the internet, which forms another big hurdle to the blind of today.
\\That's not to say that people didn't try to offer solutions to this; although it's apathetic compliance to authorities, there had been some effort to put braille under hanged labels and on the ground (Figure 0.0 demonstrates an example of braille)
\begin{figure}[h] 
	\centering
	\includegraphics[width=5cm]{example-image-A}
	\caption{Braille}
	\label{braille}
\end{figure}
\\A conundrum arises though, how could a blind person tell where that hanged label or ground protrusion is, much less know how to read it? 
\\You never know when you go blind, there's a good chance you would get it unexpectedly so you might not even know where to look to learn it if the previous paragraph also applies to said individual, in response to this, a much more respectable effort came in the form of tactile pavements, the blind shouldn't have to know anything about how to distinguish each tile, they just have to know it's presence, while they might still be able to miss it in certain cases, the authority of said place or building can just line them up in such a way that it becomes hard to miss since they also have the benefit of being generic tiles that only signal danger or caution, meaning one wouldn't have to make different tiles for each place, they just have to stick with one pattern, reducing cost of manufacturing \\and maintaining them significantly, below is a figure for an example tactile pavement.
\\  \textcolor{red} {Paragraphs above need citations to confirm their validity}
\\ \begin{figure}[h] 
	\centering
	\includegraphics[width=5cm]{example-image-A}
	\caption{Example of a tactile pavement}
	\label{tactle_pavement}
\end{figure}
\\A problem is still standing however, while they might signal potential danger or sudden elevations of ground or the edges of a track, they don't guide you anywhere so the blind is still hopelessly lost in the midst of a crowd and doesn't know which is where.
\\We've embarked on a project to change that, with the galactic rise of smartphones, odds are almost everyone is equipped with smartphones. 
\\We aim to guide them with an inexpensive, easy to set up and maintain, and with cellular, you don't even have to have access to the internet to make it, which means higher odds that remote mosques and other such buildings to use it without any fear of cost or the presence of a maintainer.

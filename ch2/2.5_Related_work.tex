\section{Related Work}

In localization research, artificial landmarks have been widely utilized due to the challenges associated with detecting and building systems based on natural landmarks. Among artificial landmarks, QR codes have garnered significant attention because of their ease of creation, low implementation costs, and the substantial amount of information they can encode.

PathFinder, proposed by Kuribayashi et al. \cite{kuribayashi2023}, is a map-less navigation system developed to assist blind individuals in navigating unfamiliar indoor environments. Unlike other systems that rely on prebuilt maps, PathFinder uses real-time detection of intersections and signs to guide users. The system includes a suitcase-shaped robot that blind users control through a handle interface, receiving audio feedback about their surroundings. PathFinder’s core navigation capabilities are powered by LiDAR for detecting intersections and image processing via a high-resolution camera to recognize directional and textual signs. Developed using a participatory design approach with blind users, the system focuses on addressing the most relevant challenges of navigating unknown spaces. In a user study involving seven blind participants, PathFinder demonstrated significant effectiveness in helping users navigate unfamiliar environments with greater confidence than when using traditional aids like canes or guide dogs. Despite requiring more user effort than map-based systems, participants appreciated PathFinder’s flexibility and adaptability to environments without prior mapping. The study concluded that PathFinder offers a valuable solution for blind individuals, particularly in situations where prebuilt maps are unavailable.

Ahmetovic et al. \cite{ahmetovic2016} presents NavCog, NavCog is a smartphone-based navigation system designed to help visually impaired individuals with turn-by-turn guidance in indoor and outdoor environments. The system uses Bluetooth Low Energy (BLE) beacons for accurate localization, employing a K-nearest neighbor (KNN) algorithm to compare current signal strengths with previously recorded beacon signal fingerprints. NavCog provides real-time auditory navigation instructions to guide users along a predetermined path and also notifies them of nearby points of interest (POI) and accessibility issues such as stairs or obstacles.

One of NavCog's key features is its ability to be easily deployed in large, complex environments without requiring significant infrastructure modifications. In a study conducted with six visually impaired participants on a university campus, NavCog successfully guided users through both familiar and unfamiliar spaces, receiving positive feedback on its navigation features. The study emphasizes NavCog’s flexibility and ease of use, making it a promising tool for visually impaired individuals navigating new environments.


The work by Fraga et al. \cite{fraga2022} presents an indoor navigation system has been developed for visually impaired individuals. This system uses QR code markers and computer vision techniques to provide real-time navigation instructions. Users scan QR codes placed in different locations to receive guidance along optimal paths. The system cross-references the scanned information with a prebuilt database. Audio feedback informs users of their current location and the next steps in the navigation process. If the user deviates from the planned route, the system recalculates the path. Importantly, this system does not require internet connectivity, making it practical for offline use. Additionally, the system includes a collision avoidance feature based on a monocular depth estimation algorithm, which predicts distances to obstacles using 2D images. Experimental results in a controlled environment demonstrated that the system accurately guided users and effectively detected obstacles with acceptable depth estimation margins. This study highlights the potential of QR code-based navigation for enhancing mobility for visually impaired individuals and suggests future integration into smartphones or AI accelerators for improved usability.


\textcolor{blue}{We should mention as a conclusion of this section how our solution will differ from them whether by accuracy, multiple tech integration, targeted audience, real world validation method, etc.}
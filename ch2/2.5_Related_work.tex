\section{Related Work}

In localization research, artificial landmarks have been widely utilized due to the challenges associated with detecting and building systems based on natural landmarks. Among artificial landmarks, QR codes have garnered significant attention because of their ease of creation, low implementation costs, and the substantial amount of information they can encode.

The work of Zhang et al. \cite{zhang2015} presents an approach for indoor mobile robot localization using QR codes arranged in a grid pattern on the ceiling. The system uses an industrial camera mounted on top of the robot, facing upward to detect these QR codes. The camera captures images of the QR codes, and a recognition algorithm processes the codes' position and orientation within the image. By leveraging the coordinates of each QR code and applying camera calibration data, the robot can accurately determine its position. This setup enables real-time, precise localization, which is crucial for efficient navigation in structured indoor environments.

Kim et al. \cite{kim2021} introduce a vision-based indoor positioning system that employs QR codes and a smartphone camera to accurately determine a user's indoor location. In this system, QR codes are placed at predefined locations and detected by the smartphone camera. The two-dimensional coordinates from the QR codes are converted into three-dimensional spatial coordinates using camera calibration techniques. By forming a quadrilateral shape from reference symbols on the QR codes, the system calculates the center of gravity to determine the user's position and orientation. Experiments demonstrated an average localization error of less than one meter, highlighting the system's high accuracy.

Lee et al. \cite{lee2015} propose a cost-effective indoor localization method for mobile robots, using QR codes as artificial landmarks. QR codes are strategically placed on the ceiling, and a smartphone mounted on the robot detects these codes to determine the robot’s position and heading direction. The positions of the QR codes are pre-stored in a database, enabling the robot to compute its real-world coordinates by processing image data captured by the smartphone's camera. The system has been experimentally validated, showing localization errors ranging from 3.2 cm to 6.55 cm, confirming its accuracy.

\textcolor{blue}{We should mention as a conclusion of this section how our solution will differ from them whether by accuracy, multiple tech integration, targeted audience, real world validation method, etc.}
\section{Landmarks}

Landmarks can be categorized as either natural or artificial. Natural landmarks are formed by nature, such as mountains, rocks, trees, or any other natural formations. While on the other hand, artificial landmarks are human-made structures such as buildings, traffic signs, statues, QR codes, and etc.. While both categorize can be used in various important fields in computer science such as robotics, and computer vision, natural landmarks are much harder to recognize, very diversified and do not have uniform shapes, difficult to customize, and they do not encode data. All of these characteristics can be opposite in the artificial landmarks.

Landmarks have a wide range of useful applications in various fields of computer science, such as robotics, localization, geographic information systems, object tracking, and a lot other useful and important applications. For example, computer vision algorithms leverage landmarks by detecting, identifying, and tracking them to determine the precise location of an object, create a 3D map of the surrounding environment, or to monitor the state of an object, and the list goes on...
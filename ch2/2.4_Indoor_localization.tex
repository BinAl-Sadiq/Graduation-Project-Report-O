\section{Indoor Localization System}

Indoor Localization is the process of calculating the precise position of an/a object/person within an indoor environment, such as a hospital, office, or shopping mall. The most popular localization technology is GPS, which is useful and works very fine in outdoor environments. But when it comes to indoor environments, GPS signals are weak or unavailable, so other technologies and techniques are used for these environments.

There are various other technologies and techniques that are used for indoor localization, such as Wi-Fi, Bluetooth, RFID, UWB, and the list goes on and on, See [1] for more information. Each one these technologies has its cons and pros, some of them rely on expensive and hard to setup hardware such as sensors, and IMU, while some other use less expensive and complicated hardware such as bluetooth beacons. While a lot of different technologies can give us precise indoor localization results, they usually don't exploit most of the visual information that exist in these environments. Thus, in this project we wanted to provide a system that makes use of these visuals which are information rich. Using only one camera, we will be able to detect landmarks to use them for localization, and extract tons of useful information from the surrounding environment.
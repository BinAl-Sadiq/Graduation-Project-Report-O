\section{Indoor Localization}

Indoor localization refers to the process of determining the precise position and orientation of an object or person within indoor environments, where traditional GPS signals are often unreliable. This technology is essential for a wide range of applications, such as navigation within large buildings, asset tracking in warehouses, and other useful and important applications.

Various technologies are employed for indoor localization, each with its own cons and pros. Wi-Fi, Bluetooth, RFID, and Ultra-Wideband (UWB) are some of the most common methods. While some systems rely on expensive and complex hardware like sensors and Inertial Measurement Units (IMUs), others use more cost-effective solutions such as Bluetooth beacons, and pose estimation using QR codes. The choice of technology often depends on the specific needs of the application, balancing factors such as accuracy, cost, and ease of implementation. For more information, see reference \cite{leitch2023}.

As indoor localization continues to evolve, new techniques and innovative approaches are emerging, including the use of QR codes for precise positioning and tracking. 

\subsection{Indoor Localization with QR Codes}

Indoor localization using QR codes is a method to determine a user’s pose. The system works by strategically placing QR codes around the space—on floors, walls, ceilings, or even hanging panels. Each QR code encodes specific positional information, allowing users to understand their location relative to these codes once they are detected.

\paragraph{The divided environment method}

The environment in this method is divided into several pieces(a piece might be a square, triangle, hexagon, and any other shape, depending on the need), each piece has its own QR Code inside of it encoding its position, whether it is put on the floor, ceiling, wall, or even in a hanging panel. See \cite{zhang2015} where they used a similar method.

For better illustration, let us assume that we have a room of 4 meters in width and height, and it is divided into 16 squares with equal sizes so each square has an area of 1m$^2$. If we customized a hat for example, that embeds a camera in its top, the user’s position will get
determined while navigating in the room wearing the hat.

Although this solution is very computationally cheap, it comes with its own
downsides. For instance, the position values are always discrete. So if we needed
a continuous and precise position we should not use this method. But actually,
this method could be very useful depending on its use, see Appendix A for more details.

\begin{figure}[h] % [h] forces the figure to be placed exactly here in the text
	\centering
	\includegraphics[width=5cm]{example-image-A}
	\caption{Grid Patter Illustration}
	\label{grid_pattern_illustration}
\end{figure}



\paragraph{Pose Estimation with QR codes}

Another approach of indoor localization involves calculating the relative position and orientation (pose) of the QR code in relation to the camera. After detecting a QR code, the camera determines its position and orientation relative to itself, and by combining this information with the known global position of the QR code, the system can estimate the user’s precise, continuous position within the space.

Although this method offers significantly higher positional accuracy, it requires more computational resources, as it involves additional steps such as camera calibration to determine intrinsic parameters. This added complexity makes it more resource-intensive compared to divided environment method, but it delivers continuous localization with greater precision. Thus this method is suitable if continues and precise values are needed. See \cite{Lucag2017} where they used a similar method.

\begin{figure}[h] % [h] forces the figure to be placed exactly here in the text
	\centering
	\includegraphics[width=5cm]{example-image-A}
	\caption{Grid Patter Illustration}
	\label{pose_estimation_illustration}
\end{figure}


In a well-designed setup, the system's effectiveness remains high, irrespective of the QR codes’ locations. By tailoring the placement and setup of QR codes to suit the environment, the system can deliver robust indoor localization, whether it prioritizes simplicity or precision.
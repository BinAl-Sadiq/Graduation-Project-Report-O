\section{Experiment Setup}

This section describes the setup of the system for conducting experiments, as part of the methodology. Once the camera calibration process is complete—an initial step that requires assistance—the system is ready for operation. The following outlines the arrangement and configuration of the components for the experiment.

\subsection{Mobile Phone Holder}
The mobile phone will be on a wearable holder to mount it at the chest, and to be directed to the user's forward direction.

\subsection{QR Code Placement}
The QR codes will be placed at the floor, each one will be 1 meter away from the other.

\subsection{Wearable Obstacle Detection System}

The wearable obstacle detection system comprises the ultrasonic sensors, vibration motors, and ESP32 microcontrollers integrated into a belt. The configuration is as follows:

\begin{itemize}
	\item The four ultrasonic sensors are placed in forward, backward, right, and left orientations on the belt to monitor the user’s surroundings.
	\item The four vibration motors are positioned correspondingly to provide tactile feedback in the same directions.
	\item Both ESP32 microcontrollers are powered by a compact power bank, with all wiring securely hidden within the belt for a clean and unobtrusive design.
\end{itemize}

The belt is worn by the user, allowing them to experience the obstacle detection and feedback system in a practical, hands-free manner.

\subsection{Back-end Infrastructure Hosting}

The backend database and web host are deployed on a remote server to ensure seamless data access and scalability during the experiment. This configuration is necessary because the system relies on two key components hosted on the remote server:

\begin{itemize}
	\item \textbf{Building Management Dashboard:} A web-based dashboard allows building managers to efficiently update, add, or modify QR code-related data, including global positions and navigation instructions. This dashboard provides centralized control over building layouts, ensuring that updates are reflected across the entire system in real time.
	\item \textbf{Database:} The database stores essential information about QR codes, such as their global positions and associated instructions. It serves as the central data repository, accessed by both the mobile application and the building management dashboard to retrieve or update QR-related data.
\end{itemize}

This remote hosting approach ensures seamless integration between the components, providing high availability and reliability. It also supports centralized management, allowing building managers to make updates remotely without requiring changes to local hardware or application configurations.


\subsection{Power Supply and Connectivity}

The system relies on portable power sources to maintain functionality throughout the experiment:
\begin{itemize}
	\item The mobile phone operates independently on its internal battery.
	\item The ESP32 microcontrollers are powered by a compact power bank attached to the wearable belt.
	\item Bluetooth Low Energy (BLE) ensures seamless communication between the ESP32 devices and the mobile application.
	\item The mobile application connects to the remote server Wi-Fi access point mobile data for backend interactions.
\end{itemize}

This setup guarantees portability, ease of use, and robust connectivity for real-world testing and experimentation.


This configuration ensures that the components are aligned and integrated effectively, enabling accurate testing and evaluation of the system in a controlled environment. The setup is designed to reflect a real-world use case as closely as possible, while allowing for flexibility and adjustments as needed during the experiment.

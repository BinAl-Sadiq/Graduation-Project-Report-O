Here’s the rewritten version focusing solely on the mobile device:

---

\section{Mobile Application}

The Mobile Application serves as the central component for visually impaired users, providing essential functionalities for navigation assistance in indoor environments. Designed with accessibility in mind, the application independently performs core tasks of localization, guidance, and real-time feedback, eliminating the need for external devices. This self-contained design ensures accurate and immediate navigation support.

The application’s functionality includes:

\begin{itemize}
	\item \textbf{QR Code Scanning and Camera Feed Processing:} The mobile application utilizes the phone’s built-in camera to capture real-time video feeds. It detects and decodes QR codes within the feed, using each code’s unique ID to fetch corresponding navigation instructions from the backend database.
	
	\item \textbf{Localization and Guidance System Implementation:} The application performs core localization and guidance tasks by analyzing data from QR codes and determining the user’s position within the environment. Based on this information, it generates tailored guidance instructions to assist users in navigating indoor spaces.
	
	\item \textbf{Backend Connectivity:} The application connects to a backend server to retrieve necessary data, such as navigation instructions and location updates. This connection allows the app to pull the latest guidance information, ensuring users receive accurate and up-to-date directions.
	
	\item \textbf{Text-to-Speech Audio Integration:} The application converts navigation instructions into audio format using a text-to-speech (TTS) module. This audio output is generated in real time, ensuring users receive immediate verbal feedback while navigating.
	
	\item \textbf{Objects Avoidance Alerts:} 
	
	\item \textbf{User-Friendly Interface:} The app’s accessible interface simplifies interaction for visually impaired users, allowing them to focus on their surroundings while receiving timely audio and vibration feedback.
	
\end{itemize}

This integrated design ensures that visually impaired users can navigate indoor environments confidently and independently. The mobile application functions as a standalone solution, coordinating all aspects of localization, guidance, backend data, camera feed processing, text-to-speech audio output, and proximity-based alerts to provide seamless and responsive navigation assistance.
\section{Mobile Application}

The Mobile Application serves as the central component for visually impaired users, providing essential functionalities for navigation assistance in indoor environments. Designed with accessibility in mind, the application performs core tasks of localization, guidance, and real-time feedback, integrating seamlessly with backend services and onboard device features. The application is designed to use the mobile device's native camera, speakers, and microphone, ensuring simplicity and ease of use.

The application’s functionality includes:

\begin{itemize}
	\item \textbf{QR Code Scanning and Camera Processing:} The mobile application uses the phone's integrated camera to capture frames and detect QR codes in real time. Upon detecting a QR code, the app decodes its unique ID and retrieves corresponding navigation instructions from the backend database.
	
	\item \textbf{Localization and Guidance System Implementation:} The mobile application determines the user’s position by analyzing the data from detected QR codes. Based on this information, the app provides step-by-step navigation guidance to assist users in navigating indoor spaces effectively.
	
	\item \textbf{Backend Connectivity:} The application connects to a backend server to retrieve necessary data, such as navigation instructions and location updates. This ensures users receive accurate and up-to-date directions in real time.
	
	\item \textbf{Objects Avoidance Alerts:} The application integrates with an obstacle detection system that uses ultrasonic sensors. Alerts are generated as vibration feedback when nearby obstacles are detected, enabling users to avoid collisions and navigate safely.
	
	\item \textbf{User-Friendly Interface:} The application interface is designed to be accessible for visually impaired users, incorporating features like TalkBack compatibility. The app provides audio feedback for navigation and tactile feedback for alerts. It ensures intuitive interaction by using larger buttons, high-contrast visual elements, and minimal manual input requirements. All critical features are accessible via simple touch-based navigation, supporting ease of use for individuals with visual impairments.
	
\end{itemize}

This design ensures that visually impaired users can navigate indoor environments confidently and independently. The mobile application serves as the core processing unit, coordinating localization, guidance, backend connectivity, and proximity-based vibration alerts, while leveraging accessible design principles to maximize usability.

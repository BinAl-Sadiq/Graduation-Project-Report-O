\section{QR Code Placement \& Camera Orientation}
Our localization implementation(see \ref{Localization System Methodology}) can calculate the users' pose despite of how the QR code is placed, as long the camera can see it. For clarifying, the codes can be placed at walls, ceilings, hanging panels, and floors and the localization system will still manage to calculate the pose at all of these positions. This is because the localization system can handle the different orientations between the camera and the QR code perfectly. So technically speaking, as long as the camera can detect the QR codes correctly, the system will work fine.

This gives the freedom of placing the QR code and camera at arbitrary orientations as long as the QR code can still be detected. But this actually can lead to a problem. For example, if a user faced his/her camera towards the ceiling, the user might enter a building where his/her camera can detect the QR codes fine because they are at the ceiling. But this same user's camera might not be able to detect the QR codes at different buildings if their QR codes where placed at the floors or walls.

The later problem raise the need of standard positions and orientations for the QR codes and cameras. Thus, we recommend a standard where the QR codes are placed at walls and hanging panels, and the cameras should be faced at a direction normal to walls.
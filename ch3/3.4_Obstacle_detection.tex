\section{Obstacle Detection}
\label{Obstacle Detection Methodology Section}

The obstacle detection system is designed to enhance the safety and navigation capabilities of visually impaired users by alerting them to nearby obstacles in real time. This system combines ultrasonic sensors, vibration motors, ESP32 MCUs, and a mobile application, enabling a seamless flow of information between hardware components and the app. By utilizing the capabilities of mobile phones, the system can also be expanded to include additional hardware in the future.

As described below, the system is composed of 8 sensors of 2 different types, each connected to a CPU running on an MCU. This setup illustrates how easily the system can be integrated with various hardware. This system can be disabled from the settings on the mobile App, giving users the flexibility of using the App without any other external hardware.

\subsection{System Overview}

The obstacle detection system comprises the following components:

\begin{itemize}
	\item Four ultrasonic sensors, each facing a specific direction: forward, backward, right, and left.
	\item Four vibration motors, placed in corresponding positions: forward, backward, right, and left.
	\item Two MCUs, one for the sensors and the other for the motors.
	\item A mobile application to process the sensor data and control vibration feedback.
\end{itemize}

\subsection{Ultrasonic Sensor Data Collection}

The ultrasonic sensors are positioned to monitor one direction each (forward, backward, right, and left). These sensors are connected to an ESP32 MCU, which measures the time it takes for an ultrasonic pulse to travel to an obstacle and reflect back. This raw time-of-flight data is sent directly to the mobile application for processing using the Bluetooth capabilities of the ESP32.

This approach ensures that the mobile phone can leverage its processing power to calculate distances and perform additional analyses if required.

\subsection{Data Processing and Mobile Application}

Upon receiving the raw time-of-flight data from the MCU, the mobile application processes the data to compute the distance to obstacles in each direction. The distance is calculated using the following equation:

\[
d = \frac{v \cdot t}{2}
\]

Where:
\begin{itemize}
	\item \(d\): Distance to the obstacle (in meters).
	\item \(v\): Speed of sound in air (approximately \(343 \, \text{m/s}\)).
	\item \(t\): Round-trip time of the ultrasonic pulse (in seconds).
\end{itemize}

The division by 2 accounts for the fact that the pulse travels to the obstacle and back.

The application determines whether the obstacle meets the actionable conditions. An obstacle is considered actionable if:
\begin{enumerate}
	\item The distance is less than 2 meters (\(d < 2\)).
	\item The obstacle has been continuously within the 2-meter range for at least 2 seconds.
\end{enumerate}

This persistence check ensures that transient obstacles do not trigger the vibration feedback unnecessarily.

If both conditions are met, the application calculates the intensity level for the corresponding vibration motors. The intensity \(I\) is adjusted based on the vibration motor's operating voltage threshold of 2.7V, as follows:

\[
I = 
\begin{cases} 
	2.7 + (0.6 \cdot (1 - \frac{d}{2})), & \text{if } d \leq 2 \text{ and persistence} \geq 2 \text{ seconds} \\ 
	0, & \text{otherwise.}
\end{cases}
\]

Where:
\begin{itemize}
	\item \(I\): Intensity of vibration (ranges from 2.7 to 3.3V for actionable distances).
	\item \(d\): Distance to the obstacle (in meters).
	\item \(3.3\): Maximum vibration intensity.
	\item \(2.7\): Minimum voltage required for motor activation.
\end{itemize}

This ensures:
\begin{itemize}
	\item When the obstacle is at the minimum measurable distance of \(d = 0.03 \, \text{m}\), the vibration intensity is at its maximum: \(I = 3.3\).
	\item When the obstacle is at \(d = 2 \, \text{m}\), the vibration intensity is \(I = 2.7\), which is just enough to activate the vibration motor.
	\item When \(d > 2 \, \text{m}\) or the persistence condition is not met, no voltage is applied (\(I = 0\)), and the motor remains idle.
\end{itemize}

The application sends these calculated intensity values to the MCU connected to the vibration motors through Bluetooth for tactile feedback.

\subsection{Vibration Feedback Mechanism}

The MCU for the vibration motors receives the calculated intensity values, which are mapped to an analog voltage ranging from 2.7 to 3.3V using Pulse Width Modulation (PWM). The voltage controls the vibration intensity, ensuring a proportional response based on the proximity and persistence of obstacles.

\[
\text{Analog Voltage} = I \quad \text{(where \(2.7 \leq I \leq 3.3\))}.
\]

Using PWM, the MCU controls the vibration motors as follows:
\begin{itemize}
	\item If \(I = 3.3\), the PWM signal generates the maximum voltage, resulting in the strongest vibration.
	\item If \(I = 2.7\), the PWM signal generates the minimum voltage required for vibration, providing low-intensity feedback.
	\item Intermediate values of \(I\) produce proportional vibrations, allowing the user to intuitively gauge obstacle proximity.
\end{itemize}

This mechanism ensures that the vibration motors are activated only when necessary, providing clear feedback that is both meaningful and easy to interpret.

The communication channel between the mobile application and the ESP32 microcontrollers is designed to be flexible, allowing for the integration of additional hardware. For example, other types of sensors could be added to the system, with the mobile leveraging its processing power to analyze and act on the new data.

\section{Evaluation Metrics}
\label{sec:evaluation_metrics}

To ensure a rigorous assessment of both the localization subsystem and the overall wearable navigation system, specific evaluation metrics were defined and applied consistently throughout all experiments. These metrics were selected to quantitatively and qualitatively assess both technical performance and user experience.

\subsection{Localization System Metrics}
The technical performance of the localization system was evaluated using the following quantitative metrics:
\begin{itemize}
	\item \textbf{Mean Localization Error:} The average Euclidean distance between the estimated and actual user positions across all reference points.
	\item \textbf{Maximum Localization Error:} The highest observed localization error during testing.
	\item \textbf{Standard Deviation:} The variation in localization error across the different test points.
	\item \textbf{Processing Time:} The average time required for QR code detection and pose estimation per scan.
\end{itemize}

\subsection{Overall System Usability Metrics}
To assess the effectiveness and usability of the system from the user perspective, a 5-point Likert scale survey was conducted, evaluating the following dimensions:
\begin{itemize}
	\item \textbf{Confidence in Navigation:} The participant’s perceived independence and confidence while navigating.
	\item \textbf{Obstacle Awareness:} Ability to detect and avoid physical obstacles.
	\item \textbf{Instruction Clarity:} The clarity and helpfulness of the audio guidance provided.
	\item \textbf{Comfort of Use:} General comfort and usability of the wearable system.
	\item \textbf{Willingness to Recommend:} Likelihood of recommending the system to other individuals with visual impairments.
\end{itemize}

All metrics and calculation methods are described here to ensure transparency and reproducibility of the evaluation process.


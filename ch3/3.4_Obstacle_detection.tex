\section{Obstacle Detection}
\label{Obstacle Detection Methodology Section}

The obstacle detection system is designed to enhance the safety and navigation capabilities of visually impaired users by providing real-time alerts for nearby obstacles. The design focuses on simplicity, reliability, and user autonomy by performing all processing on the ESP32 microcontroller and supporting straightforward configuration through the mobile application only at startup.

\subsection{System Overview}

The obstacle detection system comprises the following components:

\begin{itemize}
	\item Two ultrasonic sensors, each facing one direction: forward and backward.
	\item Two vibration motors, placed to correspond with the forward and backward directions.
	\item One ESP32 microcontroller, responsible for both sensor data processing and motor control.
	\item A mobile application, used solely for configuring the detection range during device startup.
\end{itemize}

This setup minimizes hardware complexity while maintaining effective functionality for safe and independent navigation.

\subsection{Sensor Data Processing and Workflow}

At startup, the ESP32 microcontroller enables a configuration window of 15 seconds. During this time, the user may adjust the obstacle detection range via the mobile application. If the user does not update the range, the ESP32 automatically loads the last configured value from its flash memory.

After the configuration window, the ESP32 continuously monitors the environment using the ultrasonic sensors. For each direction (forward and backward), the microcontroller calculates the distance to any detected object based on the time-of-flight measurement of ultrasonic pulses:

\[
d = \frac{v \cdot t}{2}
\]

where:
\begin{itemize}
	\item $d$: Distance to the obstacle (in meters).
	\item $v$: Speed of sound in air (approximately $343~\mathrm{m/s}$).
	\item $t$: Round-trip time of the ultrasonic pulse (in seconds).
\end{itemize}

An obstacle is considered actionable if:
\begin{enumerate}
	\item The distance $d$ is less than the configured threshold (default or user-set value).
	\item The obstacle remains within this range continuously for at least 2 seconds.
\end{enumerate}

This persistence requirement ensures that only sustained obstacles trigger a warning, minimizing false positives from transient objects.

\subsection{Vibration Feedback Mechanism}

Once both conditions are met for either direction, the ESP32 activates the corresponding vibration motor to provide haptic feedback. The vibration intensity is fixed—either on or off. The vibration continues as long as the obstacle remains within the specified range, ceasing once the path is clear.



\section{Evaluation Metrics}
\label{sec:evaluation_metrics}

To ensure a rigorous assessment of both the localization subsystem and the overall wearable navigation system, specific evaluation metrics were defined and applied consistently throughout all experiments. These metrics were selected to quantitatively and qualitatively assess both technical performance and user experience.

\subsection{Localization System Metrics}
The technical performance of the localization system was evaluated using the following quantitative metrics:
\begin{itemize}
	\item \textbf{Mean Absolute Error} 
	\item \textbf{Standard Deviation} 
\end{itemize}

\subsection{Overall System Usability Metrics}
To assess the effectiveness and usability of the system from the user perspective, a 5-point Likert scale survey was conducted, evaluating the following dimensions:
\begin{itemize}
	\item \textbf{Confidence in Navigation:} The participant’s perceived independence and confidence while navigating.
	\item \textbf{Obstacle Awareness:} Ability to detect and avoid physical obstacles.
	\item \textbf{Instruction Clarity:} The clarity and helpfulness of the audio guidance provided.
	\item \textbf{Comfort of Use:} General comfort and usability of the wearable system.
	\item \textbf{Willingness to Recommend:} Likelihood of recommending the system to other individuals with visual impairments.
\end{itemize}

All metrics and calculation methods are described here to ensure transparency and reproducibility of the evaluation process.

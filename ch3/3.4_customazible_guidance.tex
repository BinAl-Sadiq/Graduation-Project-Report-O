\section{Customizable Guidance}

The Customizable Guidance system provides tailored navigation assistance to visually impaired users, implemented directly within the mobile application. This system enables the mobile app to retrieve specific navigation instructions from a central database upon scanning a QR code offering personalized guidance to the users, such as guiding users following paths to their destinations, and reading to users unique set of instructions/information based on their locations, allowing users to receive context-aware guidance as they move throughout the building.

\subparagraph{Locations-based instructions/information}
\begin{itemize}
	\item \textbf{QR Code Scanning:} When a user scans a QR code, the application identifies the QR code's unique ID.
	\item \textbf{Instruction Retrieval:} Based on the ID, the system fetches the relevant instructions from a remote database.
	\item \textbf{Audio or Visual Feedback:} The retrieved instructions can be provided as audio guidance, but the system can also present information in other forms, depending on user preferences and context.
	\item \textbf{Voice Commands:} Users can interact with the system using voice commands through the mobile application. For instance, they can request notifications for specific scenarios, such as “notify me when someone with a red shirt is nearby.”
\end{itemize}

\subparagraph{Destination path guidance}
Since the user's exact position and rotation are known because of the localization system(See \ref{Localization System Methodology}), the guidance system is able to determine the path between the user and the desired destination. The way this work is as follows:
\begin{enumerate}
	\item The user asks the system using a microphone to guide him/her navigating to a certain destination.
	\item The system will search the database for any that matches the decryption.
	\item If the destination was determined, the path is fragmented into several milestones, starting from the user, and ending at the destination.
	\item Calculate the distance and angle between the user current and the next milestone:
	\begin{equation}
		difference = milestone\_position - user\_global\_position
	\nonumber\end{equation}
	\begin{equation}
		distance = \sqrt{difference.x^2 + difference.y^2}
	\nonumber\end{equation}
	\begin{equation}
		angle = tan^{-1}(\frac{difference.y}{difference.x})
	\nonumber\end{equation}
	\item Based on the later calculations, the guidance system will inform the user - using a speaker - to move in a certain direction for a specific distance.
	\item After reaching the milestone, if the next milestone was on another floor, the system will inform the user that he/she is standing in front of the stairs/elevator and ask him/her to go to the correct floor.
	\item If the user did not reach the destination, steps 3 to 5 will be repeated.
\end{enumerate}

This approach ensures that users receive accurate and timely navigation information tailored to their environment while also allowing for interactive communication with the system.
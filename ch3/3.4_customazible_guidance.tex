\section{Customizable Guidance}
\label{Customizable Guidance Methodology Section}

The Customizable Guidance system provides navigation assistance to visually impaired users through the mobile application. By utilizing location-specific data, the system retrieves tailored navigation instructions from a central database when a QR code is scanned. This allows users to receive context-aware guidance, such as directions to their destination or specific information relevant to their current location. The system ensures an intuitive and accessible experience as users move throughout a building.

\subsection{Location-Based Instructions and Information}

The Customizable Guidance system operates through the following steps:

\begin{itemize}
	\item \textbf{QR Code Scanning:} When a user scans a QR code using the mobile application's camera, the system identifies the QR code's unique ID.
	\item \textbf{Instruction Retrieval:} The unique ID is used to fetch specific instructions or information associated with the QR code's location.
	\item \textbf{Accessible Presentation:} The retrieved instructions are displayed on the mobile screen and are accessible through assistive features like TalkBack, ensuring visually impaired users can hear the information read aloud.
\end{itemize}

This approach allows visually impaired users to navigate buildings with ease, offering real-time, location-specific assistance that is both customizable and intuitive.


\subparagraph{Destination Path Guidance}

Since the user's exact position and rotation are known through the localization system (See \ref{Localization System Methodology}), the guidance system can determine the optimal path between the user and their desired destination. The system works as follows:

\begin{enumerate}
	\item The user selects a destination from a displayed list of zones within the building. The list is navigable and selectable using TalkBack, allowing visually impaired users to choose their destination as they typically interact with their phones.
	\item Once a destination is selected, the system searches the database for its corresponding location and determines the optimal path.
	\item The path is fragmented into several milestones, starting from the user's current position and ending at the selected destination.
	\item For each milestone, the system calculates the distance and angle between the user's current position and the next milestone:
	\begin{equation}
		\text{difference} = \text{milestone\_position} - \text{user\_global\_position}
		\nonumber\end{equation}
	\begin{equation}
		\text{distance} = \sqrt{\text{difference.x}^2 + \text{difference.y}^2}
		\nonumber\end{equation}
	\begin{equation}
		\text{angle} = \tan^{-1}\left(\frac{\text{difference.y}}{\text{difference.x}}\right)
		\nonumber\end{equation}
	\item Based on these calculations, the guidance system informs the user, via TalkBack, to move in a specific direction and for a specified distance.
	\item If the next milestone is on a different floor, the system informs the user that they are standing in front of the stairs or elevator and instructs them to proceed to the correct floor.
	\item The system continues to guide the user by repeating steps 3 to 5 until the destination is reached.
\end{enumerate}

This method ensures a seamless and accessible experience for visually impaired users by leveraging their familiarity with TalkBack for destination selection and providing step-by-step guidance to navigate the path effectively.


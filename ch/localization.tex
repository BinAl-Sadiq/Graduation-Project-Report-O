\section{Localization System Evaluation}

\subsection{Experiment Setup}

To evaluate the proposed localization system, a QR code was mounted on a wall, and the user's pose relative to it was estimated using the system at different angles, distances, and lighting conditions.

\textbf{Camera Used:}\\
The rear camera of the Honor ALI-NX1 smartphone was used for all experiments.

\textbf{Camera Calibration:}\\
The camera was calibrated using our implementation. Fifteen images of a calibration pattern were taken at random angles and distances. The reprojection error was found to be 0.0737, demonstrating excellent calibration even with random sampling.

\textbf{QR Code Setup:}\\
The QR code was placed on a wall at a fixed height to simplify measurements (see Figure~\ref{Localization-Experiment-QR-Setup}).

\begin{figure}[h!]
	\centering
	\includegraphics[width=0.7\linewidth]{assets/ch4/QR on a wall.jpg}
	\caption{A picture illustrating the QR code setup.}
	\label{Localization-Experiment-QR-Setup}
\end{figure}

\textbf{Lighting Conditions:}\\
Localization was tested under three distinct lighting conditions (see Figure~\ref{Localization-Experiment-Lighting-Conditions}):
\begin{itemize}
	\item \textbf{Perfect Lighting}
	\item \textbf{Poor Lighting}
	\item \textbf{No Lighting at All (Dark):} Here, the camera's built-in flash was used.
\end{itemize}

\begin{figure}[h!]
	\centering
	\includegraphics[width=0.7\linewidth]{assets/ch4/Three lighting conditions/Three lighting conditions.png}
	\caption{The three distinct lighting conditions: left—perfect, middle—poor, right—dark.}
	\label{Localization-Experiment-Lighting-Conditions}
\end{figure}

\textbf{Measuring Tools:}\\
A tape measure was used to measure the camera's $x$ and $y$ positions relative to the QR code, allowing calculation of both angle and distance. This method achieves an accuracy of $\pm5$~mm.

\subsection{Procedure}\label{sec:procedure}
For each lighting condition, the experiment was conducted three times. The system was evaluated at distances of 1.410~m and 0.707~m, each at angles of $0.00^\circ$, $+45.00^\circ$, and $-45.00^\circ$.

\subsection{Evaluation Metrics}
The evaluation metrics applied in this experiment are described in Section~\ref{sec:evaluation_metrics} of the Methodology chapter. All results and analyses in this section are based on those metrics.

\subsection{Results}

The results for all lighting conditions are summarized in Table~\ref{table:localization_exp_res_combined}. Each row presents the estimated and real values of distance and angle for each combination of conditions.

\begin{table}[h!]
	\caption{Estimated and real distances/angles under all lighting conditions.}
	\begin{tabularx}{0.95\textwidth} { 
			| >{\raggedright\arraybackslash}X
			| >{\centering\arraybackslash}X
			| >{\centering\arraybackslash}X
			| >{\centering\arraybackslash}X
			| >{\centering\arraybackslash}X
			| >{\centering\arraybackslash}X | }
		\hline
		Lighting & Estimated Distance & Estimated Angle & Real Distance & Real Angle \\
		\hline
		Perfect  & 0.960 m & -2.40$^\circ$  & 1.410 m & 0.00$^\circ$ \\
		Perfect  & 0.989 m & +43.20$^\circ$ & 1.410 m & +45.00$^\circ$ \\
		Perfect  & 0.981 m & -42.60$^\circ$ & 1.410 m & -45.00$^\circ$ \\
		Perfect  & 0.481 m & +3.10$^\circ$  & 0.707 m & 0.00$^\circ$ \\
		Perfect  & 0.496 m & +47.30$^\circ$ & 0.707 m & +45.00$^\circ$ \\
		Perfect  & 0.476 m & -42.60$^\circ$ & 0.707 m & -45.00$^\circ$ \\
		\hline
		Poor     & 0.984 m & -3.00$^\circ$  & 1.410 m & 0.00$^\circ$ \\
		Poor     & 0.972 m & +44.50$^\circ$ & 1.410 m & +45.00$^\circ$ \\
		Poor     & 0.946 m & -45.60$^\circ$ & 1.410 m & -45.00$^\circ$ \\
		Poor     & 0.511 m & -4.30$^\circ$  & 0.707 m & 0.00$^\circ$ \\
		Poor     & 0.485 m & +43.30$^\circ$ & 0.707 m & +45.00$^\circ$ \\
		Poor     & 0.516 m & -47.80$^\circ$ & 0.707 m & -45.00$^\circ$ \\
		\hline
		Dark     & 0.992 m & +4.20$^\circ$  & 1.410 m & 0.00$^\circ$ \\
		Dark     & 1.014 m & +44.20$^\circ$ & 1.410 m & +45.00$^\circ$ \\
		Dark     & 0.998 m & -44.80$^\circ$ & 1.410 m & -45.00$^\circ$ \\
		Dark     & 0.501 m & +3.10$^\circ$  & 0.707 m & 0.00$^\circ$ \\
		Dark     & 0.500 m & +42.70$^\circ$ & 0.707 m & +45.00$^\circ$ \\
		Dark     & 0.526 m & -48.00$^\circ$   & 0.707 m & -45.00$^\circ$ \\
		\hline
	\end{tabularx}
	\label{table:localization_exp_res_combined}
\end{table}

Evaluation results, including mean absolute error and standard deviation for each lighting condition, are presented in Table~\ref{table:localization_eval}.

\begin{table}[h!]
	\caption{Evaluation results for each lighting condition.}
	\begin{tabularx}{0.95\textwidth} { 
			| >{\raggedright\arraybackslash}X 
			| >{\centering\arraybackslash}X 
			| >{\centering\arraybackslash}X 
			| >{\centering\arraybackslash}X 
			| >{\centering\arraybackslash}X | }
		\hline
		Lighting Condition & Distance Mean Abs. Error & Angle Mean Abs. Error & Distance Std. Dev. & Angle Std. Dev. \\
		\hline
		Perfect & 0.328 & 2.40    & 0.246 & 35.92 \\
		Poor    & 0.323 & 2.15   & 0.232 & 37.01 \\
		Dark    & 0.303 & 2.26 & 0.246 & 36.77 \\
		\hline
	\end{tabularx}
	\label{table:localization_eval}
\end{table}

\begin{figure}[h!]
	\centering
	\includegraphics[width=0.9\linewidth]{assets/ch4/distacneerro.pdf}
	\caption{Absolute distance error ($|\mathrm{Estimated} - \mathrm{Real}|$) for each sample under different lighting conditions. Lower values indicate higher distance accuracy.}
	\label{fig:localization_absolute_distance_error}
\end{figure}

\begin{figure}[h!]
	\centering
	\includegraphics[width=0.9\linewidth]{assets/ch4/angleerro.pdf}
	\caption{Absolute angle error ($|\mathrm{Estimated} - \mathrm{Real}|$) for each sample under different lighting conditions. Lower values indicate higher angular accuracy.}
	\label{fig:localization_absolute_angle_error}
\end{figure}

\subsection{Results Discussion}

As shown in Table~\ref{table:localization_eval}, lighting conditions had negligible impact on accuracy. Notably, the angle estimates were consistently very close to the real values, which is desirable, as precise angles are crucial for user navigation—small deviations could result in significant guidance errors. In contrast, errors in estimated distance are less critical, as the system logic mitigates their effect.

For example, if the estimated distance to a QR code is 2~m while the real value is 2.6~m (a 23\% error), the application adds this to the average distance between milestones (e.g., 3~m), reducing the overall error effect to just 10.7\%. Table~\ref{table:real_estimated_distance_error_effect} illustrates this for typical cases.

\begin{table}[h!]
	\caption{Actual impact of estimated distance errors, assuming an average milestone distance of 3~m.}
	\begin{tabularx}{0.75\textwidth} { 
			| >{\raggedright\arraybackslash}X 
			| >{\centering\arraybackslash}X 
			| >{\centering\arraybackslash}X 
			| >{\raggedleft\arraybackslash}X | }
		\hline
		Estimated Distance & Real Distance & Fake Error Effect & Real Error Effect\\
		\hline
		0.511 m & 0.707 m & 27.722\% & 5.287\%\\
		1.014 m & 1.410 m & 28.085\% & 8.98\%\\
		\hline
	\end{tabularx}
	\label{table:real_estimated_distance_error_effect}
\end{table}

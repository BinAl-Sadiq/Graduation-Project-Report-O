\section{Overall System Evaluation}

\subsection{Experiment Design}

The overall system evaluation was conducted in a multi-room indoor facility simulating a real-world public building. The space included hallways, turns, open areas, and furniture that introduced natural obstacles and navigation challenges.

\textbf{Lighting Conditions:}  
The room had stable and sufficient lighting provided by ceiling-mounted fluorescent fixtures, ensuring clear visibility for QR code detection throughout the experiment.

\textbf{Building Layout:}  
A predefined navigation route was selected that included multiple turns, obstacle zones, and doorways. The total length of the route was approximately 15 meters and included four direction changes and several static obstacles, such as tables and chairs.

\textbf{Participant Setup:}  
Six participants took part in the evaluation. Each participant was blindfolded to simulate visual impairment. The mobile phone running the navigation system was mounted on a chest harness, while vibration feedback devices were worn around the waist. Participants also wore a pair of wireless earphones to receive audio instructions.

\begin{figure}[h]
	\centering
	\includegraphics[width=0.7\linewidth]{example-image-a}
	\caption{Floor plan of the building showing the test navigation route and obstacle locations (placeholder).}
	\label{fig:building_layout}
\end{figure}

\begin{figure}[h]
	\centering
	\includegraphics[width=0.7\linewidth]{example-image-a}
	\caption{Participant navigating the route using the wearable system (placeholder).}
	\label{fig:overall_experiment_procedure}
\end{figure}

\subsection{Procedure}

Participants were randomly divided into two groups of three. The first group used the complete Mosaned system, which included QR-based localization, real-time audio guidance, and obstacle detection via vibration feedback. The second group relied on traditional navigation methods such as verbal instructions from a human assistant.

Each participant was asked to navigate independently from the starting point to the destination along the defined path. Observers recorded behavioral observations, including navigation errors, assistance required, and time taken. At the end of the experiment, participants completed a Likert-scale survey evaluating their experience.

\subsection{Evaluation Metrics}

To assess the effectiveness and usability of the system, participants rated their experience using a 5-point Likert scale across the following dimensions:

\begin{itemize}
	\item \textbf{Confidence in Navigation:} The level of independence felt during the task.
	\item \textbf{Obstacle Awareness:} Ability to detect and avoid physical obstacles.
	\item \textbf{Instruction Clarity:} The perceived clarity and helpfulness of audio instructions.
	\item \textbf{Comfort of Use:} General comfort using the wearable hardware and software interface.
	\item \textbf{Willingness to Recommend:} Likelihood of recommending the system to others with visual impairments.
\end{itemize}

\subsection{Results}

The results of the user survey are presented in Figure~\ref{fig:user_survey}, which compares average ratings between the group using the proposed system and those using traditional methods. Each bar represents a category scored from 1 (Strongly Disagree) to 5 (Strongly Agree).

\begin{figure}[h]
	\centering
	\includegraphics[width=0.7\linewidth]{example-image-a}
	\caption{Comparison of average user ratings between Mosaned system and traditional navigation (placeholder).}
	\label{fig:user_survey}
\end{figure}

Participants who used the Mosaned system reported .... (Complete based on the survey)

\section{Results}
To be obtained at the next semester.




\subsection{Localization System Evaluation Results}

The localization system was evaluated in a controlled meter indoor environment to assess its accuracy and reliability. The system estimated the user’s position based on a single QR code placed at a fixed location, using known relative angles and distances between the QR code and the camera as illustrated in the experiment setup 39.9.9


\subsubsection*{Results}
The evaluation results are summarized in Table~\ref{tab:localization_results}. The mean localization error was found to be \textbf{TBD} cm, with a maximum error of \textbf{TBD} cm. The standard deviation was \textbf{TBD}, indicating \textbf{(low/high) variation} in accuracy. The average processing time per estimation was \textbf{TBD} ms.

\begin{table}[h]
	\centering
	\caption{Localization System Error Metrics}
	\label{tab:localization_results}
	\begin{tabular}{|c|c|}
		\hline
		\textbf{Metric} & \textbf{Value} \\ 
		\hline
		Mean Localization Error (cm) & TBD \\ 
		\hline
		Maximum Localization Error (cm) & TBD \\ 
		\hline
		Standard Deviation (cm) & TBD \\ 
		\hline
		Processing Time (ms) & TBD \\ 
		\hline
	\end{tabular}
\end{table}

To further analyze accuracy, Figure~\ref{fig:error_distribution} shows the distribution of localization errors. The majority of errors were within \textbf{TBD} cm, demonstrating that the system provides \textbf{(highly/moderately) reliable positioning}. Additionally, Figure~\ref{fig:error_vs_distance} illustrates how localization error varied with the user’s distance from the QR code.

\begin{figure}[h]
	\centering
	\includegraphics[width=0.8\textwidth]{example-image-a} % Replace with actual file name
	\caption{Distribution of Localization Errors}
	\label{fig:error_distribution}
\end{figure}

\begin{figure}[h]
	\centering
	\includegraphics[width=0.8\textwidth]{example-image-a} % Replace with actual file name
	\caption{Localization Error vs. Distance from QR Code}
	\label{fig:error_vs_distance}
\end{figure}

\subsubsection*{Discussion}
The results indicate that the system \textbf{(performed well/showed some limitations)} in estimating user positions based on a single QR code. The localization error remained within an acceptable range for most test cases, though deviations were observed in \textbf{(specific conditions such as extreme angles or large distances from the QR code)}.

\textbf{Potential Improvements:}
\begin{itemize}
	\item Enhancing QR code detection under varying lighting conditions.
	\item Refining pose estimation algorithms for improved accuracy.
	\item Investigating additional computational techniques to reduce localization error at larger distances.
\end{itemize}

Overall, the system demonstrated \textbf{(strong/moderate) potential} in providing accurate indoor positioning using a single reference point, with opportunities for further optimization.



\section{Discussion}
\paragraph{overall}

\paragraph{Localization System with QR code}
The results indicate that the system \textbf{(performed well/showed some limitations)} in estimating user positions based on a single QR code. The localization error remained within an acceptable range for most test cases, though deviations were observed in \textbf{(specific conditions such as extreme angles or large distances from the QR code)}.

\textbf{Potential Improvements:}
\begin{itemize}
	\item Enhancing QR code detection under varying lighting conditions.
	\item Refining pose estimation algorithms for improved accuracy.
	\item Investigating additional computational techniques to reduce localization error at larger distances.
\end{itemize}

Overall, the system demonstrated \textbf{(strong/moderate) potential} in providing accurate indoor positioning using a single reference point, with opportunities for further optimization.







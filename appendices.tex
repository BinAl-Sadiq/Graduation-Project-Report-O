\appendix % Switch to appendices mode

\chapter{}
\section{The divided environment method}

The idea of dividing an environment into smaller pieces is widely used in
some industries in different ways and purposes. One of these industries is the video game industry. A huge amount of video games divide the game world
into a chunk of squares, triangles, hexagons, and other shapes depending on the
game’s need and performance. But the difference here is that they do not use
this technique for localization since the positions for all the objects in the games
are known and stored at the RAMs already. They use the technique to categorize
the pieces such as walkable ground, water, lava, rocks, and so on. Then these
pieces are used along with their categories to find a proper path between two
points. Different path finding algorithms can be used, but the most popular and
simple one is A* algorithm. See this \cite{SebastianLague_AStar} incredible youtube tutorial made by
Sebastian Lague, that explain an implementation of the A* algorithm and how
it can be optimized.

\section{Metadata and Features of the Database}
\label{appendix:db_metadata}

The database system is composed of five main tables, each designed to manage specific elements of the Mosaned system. This section details the metadata and features associated with each table.

\begin{itemize}
	\item \textbf{BuildingAdmin:}
	\begin{itemize}
		\item \textbf{user\_id (Primary Key)}: A unique identifier for each user of the system, enabling individual management of buildings and associated data.
		\item \textbf{username}: The name chosen by the user for login purposes.
		\item \textbf{password\_hash}: A secure hash of the user's password used for authentication.
		\item \textbf{email}: The user's email address, utilized for notifications and account recovery.
		\item \textbf{created\_at}: A timestamp indicating when the user account was created.
	\end{itemize}
	
	\item \textbf{Buildings:}
	\begin{itemize}
		\item \textbf{building\_id (Primary Key)}: A unique identifier for each building within the system.
		\item \textbf{user\_id (Foreign Key)}: Links the building to the associated user in the \texttt{BuildingAdmin} table, ensuring that each building is managed by the correct user.
		\item \textbf{building\_name}: The name of the building.
		\item \textbf{location}: The physical address or description of the building's location.
		\item \textbf{instructions}: General instructions related to the building, providing contextual information for navigation and management.
	\end{itemize}
	
	\item \textbf{Floors:}
	\begin{itemize}
		\item \textbf{floor\_id (Primary Key)}: A unique identifier for each floor within a building.
		\item \textbf{building\_id (Foreign Key)}: Associates the floor with its corresponding building, establishing a clear hierarchical structure.
		\item \textbf{floor\_number}: Indicates the number of the floor (e.g., 1 for the first floor, 2 for the second, etc.).
		\item \textbf{instructions}: Specific instructions related to that floor, assisting users with effective navigation.
	\end{itemize}
	
	\item \textbf{Sections:}
	\begin{itemize}
		\item \textbf{section\_id (Primary Key)}: A unique identifier for each section on a floor.
		\item \textbf{floor\_id (Foreign Key)}: Links the section to its respective floor, maintaining the organizational hierarchy.
		\item \textbf{section\_name}: The name or description of the section within the floor.
		\item \textbf{instructions}: Instructions specific to the section, enhancing the user’s ability to navigate the area.
	\end{itemize}
	
	\item \textbf{QR\_Codes:}
	\begin{itemize}
		\item \textbf{qr\_id (Primary Key)}: A unique identifier for each QR code, ensuring no duplicates exist within the system.
		\item \textbf{section\_id (Foreign Key)}: Connects the QR code to its specific section.
		\item \textbf{global\_position\_x}: The X-coordinate representing the global position of the QR code.
		\item \textbf{global\_position\_y}: The Y-coordinate indicating the QR code's position, further aiding in navigation.
		\item \textbf{instructions}: Specific instructions directly linked to the QR code, providing targeted guidance for users when the QR code is scanned.
	\end{itemize}
\end{itemize}

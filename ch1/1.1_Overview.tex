\section{Overview}
Navigation in indoor areas presents considerable difficulties for those with visual impairments. Conventional GPS systems, while proficient in outside navigation, are inadequate for indoor environments due to diminished satellite signals and the intricate designs of spaces such as retail malls, hospitals, and educational institutions. Researchers have investigated alternate ways, including the use of artificial landmarks, to facilitate indoor navigation.

Artificial landmarks, including QR codes, offer a cost-effective and readily implementable alternative for localisation and navigation in interior settings. These landmarks are identifiable by cameras or sensors, enabling visually challenged individuals to obtain audio or tactile feedback for navigational assistance. This project proposes a QR code-based navigation system to ensure precise localisation and dependable obstacle detection for those with visual impairments.
This project aim to develop a low-cost indoor navigation system that will combine accurate indoor localization, reliable obstacle detection, and customizable, for indoor environment using (…Tools and Software and QR code) to help with impairments visuality. 


